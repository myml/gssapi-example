\documentclass[a4paper]{scrartcl}
\usepackage{xltxtra}
\usepackage{plantuml}
\usepackage{listings}
\usepackage[UTF8,noindent]{ctex}
\usepackage[colorlinks,linkcolor=blue]{hyperref}

% \setmainfont[Mapping=tex-text]{WenQuanYi Micro Hei}
\begin{document}
\tableofcontents
\clearpage
\section{Kerberos、GSSAPI和SPNEGO}
\begin{plantuml}
    @startuml
    left to right direction
    (GSSAPI) <|.. (Kerberos): 实现
    (SPNEGO) --> (GSSAPI): 依赖
    @enduml
\end{plantuml}
\\
GSSAPI是一种安全机制,定义了通用的安全接口规范。应用可以使用GSSAPI创建安全的上下文进行身份验证,而不用关心安全的实现细节。
\\
\\
Kerberos是一个协议,并实现了GSSAPI的接口,也代指其具体实现服务程序MIT Kerberos(krb5)。
\\
支持的应用有ssh、pam、nfs、curl、postgresql、mysql、hadoop、chrome、firfox、nginx、ad域等
\\
Kerberos协议允许使用对称密钥加密在非安全的传输层进行双向身份验证,kerberos需要一个可信的第三方密钥分发中心(KDC)。
\\\\
\begin{plantuml}
@startuml
主体A -> KDC: 我是主体A,想访问主体B
主体A <-- KDC: 给你一个会话密钥,用主体A的密钥加密了 \n 给你一个授权票据,用主体B的密钥加密了
note right
验证主体A的身份,同时也能证明自己的身份
授权票据给主体B的,主体A不需要知道内容
end note
主体A -> 主体A: 用自己的密钥(密码的哈希)解密会话密钥
主体A -> 主体B: 会话密钥加密{我是主体A},这是KDC给我的授权票据
主体B -> 主体B: 用自己的密钥解密授权票据,从票据中获取会话密钥 \n 用会话密钥解密主体A的消息,验证主体A的身份
主体A <- 主体B: 会话密钥加密{我是主体B}
note right
证明自己的身份
end note
@end
\end{plantuml}
\\
SPNEGO是一种认证协商的协议,用于自动选择认证方式并简化了认证流程,通常选择项是Kerberos和NTLM,可在浏览器中基于HTTP协议进行验证。
\clearpage
\section{kerberos协议}
KDC包含认证服务和授权服务,他们知道自己的密钥,也知道其他主体的密钥,其他主体只需要知道自己的密钥。\\\\
\begin{plantuml}
    @startuml
    box "KDC"
    participant 认证服务 order 2
    participant 授权服务 order 3
    end box
    participant 客户端 order 1
    participant 第三方服务 order 0
    autonumber
    客户端 -> 认证服务: 请求认证
    note right
    附带用户ID(用户名或邮箱)
    end note
    客户端 <-- 认证服务: 授权会话密钥
    note right
    一个授权会话密钥K1(使用用户密钥加密)
    end note
    客户端 <-- 认证服务: 授权会话票据
    note right
    一个授权票据P1(授权密钥加密)
    包含用户ID等认证信息,还有流程2中的K1
    end note
    客户端 -> 客户端: 解密
    note right
    使用用户密钥解密得到授权会话密钥K1
    end note
    客户端 -> 授权服务: 请求授权
    note right
    用户唯一标识(使用授权会话密钥K1加密)
    流程3中的票据P1
    end note
    授权服务 -> 授权服务: 解密
    note left
    使用授权密钥解密P1
    取出会话密钥K1再解密出用户标识
    end note
    客户端 <-- 授权服务: 第三方服务会话密钥
    note right
    一个服务会话密钥K2(授权会话密钥加密)
    end note
    客户端 <-- 授权服务: 第三方服务会话票据
    note right
    一个授权票据P2(第三方服务密钥加密)
    包含用户ID等认证信息,还有流程7中的K2
    end note
    客户端 -> 第三方服务: 登录
    第三方服务 -> 第三方服务: 解密
    note right
    使用服务密钥解密P2,取出服务会话密钥K2
    再服务会话密钥K2解密出用户标识
    end note
    客户端 <-- 第三方服务: 登录成功
    
    @enduml
\end{plantuml}    
\\\\
经过授权认证后,客户端信任第三方服务,第三方服务也信任客户端。
使用对称加密存在密码爆破的问题,认证授权过程中加入临时生成的会话密钥减少被攻击的可能性,会话密钥有一段有效期,也能提升性能。
\clearpage

\section{kerberos优点和缺点}
\begin{itemize}
    \item 密钥不会被窃听\\
    密钥不经过网络传播,票据对应到每个主机,内部使用哈希加盐的方式存储密钥。
    \item 配置复杂\\
    服务端配置较为复杂,客户端也需要进行配置。
    \item 单点登录\\
    只需输入密码验证一次,任何使用geeapi的应用都可以自动登录。
    \item 单点故障\\
    中心话的方式导致必须保证可用性,否则所有相关应用都会收到影响
    \item 兼容性广泛\\
    可应用于各个基础设施,甚至是操作系统,而不仅仅是web。
    \item 很旧的协议\\
    九十年代的协议,广泛的应用,意味着更大的攻击范围
\end{itemize}
\section{Kerberos常用操作}
\begin{lstlisting}[language={bash}]
# 登录
kinit admin/admin@EXAMPLE.COM
# 查看票据
klist
# 登出
kdestroy
# 进入管理员界面
kadmin
    # 创建主体(使用随机密码)
    addprinc -randkey HTTP/b.example.com@EXAMPLE.COM
    # 查看主体列表
    listprincs
    # 导出到密钥表
    ktadd -k b.keytab HTTP/b.example.com@EXAMPLE.COM

\end{lstlisting}

\clearpage
\section{SPNEGO编程}
SPNEGO用于C/S在未知对方支持的身份认证协议的情况下,协商认证协议使用的,有微软制定,得到所有主流浏览器支持,安全认证协议一般选项有NTLM和Kerberos,NTLM因为加密算法太弱,已不推荐使用。
\\
SPNEGO使用gssapi接口进行安全认证,并通过http头传递认证token。
\\
网站对接SPNEGO不需要和KDC通信,但需要配置从KDC导出的密钥,用于解密授权token。
\\
SPNEGO提供单点登录但不提供传输加密,需使用https保护连接。
\subsection{使用范例}
\begin{lstlisting}[language={bash}]
curl --negotiate -vv -u : http://b.example.com:8080
google-chrome-stable --auth-server-whitelist="*example.com"
\end{lstlisting}

\begin{plantuml}
    @startuml
        浏览器 -> 网站: 访问
        浏览器 <-- 网站: 401
        note right
        头信息
        Www-Authenticate: Negotiate
        end note
        浏览器 -> kerberos: 申请授权
        note left
        gssapi接口
        end note
        return
        浏览器 -> 网站: 访问
        note right
        头信息
        Authorization: Negotiate base64(token)
        end note
        网站 -> 网站: 验证token并获取授权信息
        return 认证通过
    @enduml
\end{plantuml}
\clearpage
\section{GSSAPI编程}
\subsection{GSSAPI客户端开发}

\begin{itemize}
    \item 初始环境,配置krb5 user
    \item 使用kinit命令进行登录
    \item 使用 gss\_import\_name 函数导入服务名
    \item 使用gss\_init\_sec\_context初始化上下文,获取output token
    \item 将output token发送到服务端,并读取返回的input token
    \item 再次执行gss\_init\_sec\_context并传入input token
\end{itemize}

\subsection{GSSAPI客户端开发}
\begin{itemize}
    \item 配置密钥路径到环境遍历
    \item 接收客户端发送的input token
    \item 使用gss\_accept\_sec\_context初始化上下文,并传入input token
    \item 将output token发送到客户端
\end{itemize}
\begin{plantuml}
    @startuml
    客户端 -> 客户端: gss_init_sec_context
    客户端 -> 服务端: 发送output token
    服务端 -> 服务端: gss_accept_sec_context
    服务端 --> 客户端: 发送output token
    客户端 -> 客户端: gss_init_sec_context
    @enduml
\end{plantuml}
\clearpage
\section{参考链接}
\href{https://tools.ietf.org/html/rfc2743}{GSSAPI规范}
\\
\href{https://tools.ietf.org/html/rfc4178}{SPNEGO规范}
\\
\href{https://tools.ietf.org/html/rfc4121}{Kerberos规范}
\\
\href{https://github.com/pbrezina/gssapi-auth}{GSSAPI编程示例(c语言)}
\\
\href{https://github.com/myml/gssapi-example}{GSSAPI编程示例(cgo)}
\\
\href{https://github.com/myml/gssapi-example}{SPNEGO编程示例(golang)}
\clearpage
\end{document}
